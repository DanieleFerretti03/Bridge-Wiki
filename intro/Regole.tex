\section{Le regole di questo gioco}

\subsection{Lo scopo}
In generale, come tanti altri giochi, fare più punti degli avversari.
Un po' generico adesso, ma più avanti si vedrà più nel dettagli come ottenere questi punti.
In un torneo, è quello di ottenere per più volte il punteggio migliore nelle singole partite.
Fai caso che a differenza di altri giochi, il bridge si gioca in coppia e non soli soletti.
I giocatori prendono i nomi dei punti cardinali, Nord gioca con Sud e Ovest gioca con Est.

\subsection{Fasi del gioco}
Dividiamo il gioco in due fasi, la prima detta \textit{licita}, la seconda detta \textit{gioco della carta}.
Al termine del gioco si segnano i punti effettuati durante la partita.

\subsubsection{Licita}
Il mazzo andrà mischiato, e le carte distribuite ai quattro giocatori.
Il giocatore che distribuisce le carte è fatto a turno.
Se siete in un torneo le carte vengono date dal giocatore indicato dal board.

\paragraph{Cos'è il board?}
Il board è una sorta di astuccio che tiene le carte dei quattro giocatori.
Ogni reparto dei questo astuccio è distinto dalle lettere dei punti cardinali.
A cosa serve? Permette di far giocare altre coppie con le stesse carte.

Questa fase la possiamo vedere come un'asta.
Lo scopo ultimo di quest'asta è quello di ottenere il diritto di giocare come attaccante/giocante.

\paragraph{Come posso fare a partecipare all'asta?}
Si può fare una 'puntata' tramite l'uso dei cartellini presenti nel 'bidding box'.
Quando un giocatore poggia sul tavolo uno dei cartellini facendo una 'puntata'
si dice che sta licitando.
Attenzione, questa fase di gioco si svolgerà in silenzio.

\paragraph{Chi inizia a dichiarare?}
Inizia a dichiarare il giocatore con la "D" sopra il board.
Questo giocatore è lo stesso che dovrà mischiare e dare le carte.
La fase di licita si svolge in senso orario a partire da questo giocatore.
Fino a che un giocatore non ha deposto un cartellino sul tavolo i successivi devono attendere.

\paragraph{Cosa rappresenta questo contratto?}
È un impegno ad effettuare nella fase di gioco un certo numero di prese con le carte.

\paragraph{Cosa significa dichiarare?}
Significa posare sul tavolo un cartellino preso dal bidding box.

\paragraph{Quando termina questa fase?}
La licita può terminare in due modi:
\begin{itemize}
  \item Se i giocatori passano tutti (4 passo di fila):
    nessuno dei giocatori ha reputato la sua mano in grado di essere giocata.
    Questo comporta la rismazzata delle carte.
    
    \begin{center}
    \begin{auction}
      {}, {p}, {p}, {p}, 
      {p}
    \end{auction}
    \end{center}

  \item A seguito di una dichiarazione e se i tre giocatori successivi dichiarano "passo".
    In questo caso, l'ultimo contratto licitato sarà usato. 
    \begin{center} 
    \begin{auction}
      {}, {p}, 1S, p, 
      p, p
    \end{auction}
    \end{center}
\end{itemize}
Nota: nei due esempi citati si suppone che il giocatore Nord sia il giocatore che ha distribuito le carte.
Ovvero, quello con la "D" nel board.

\subsubsection{Gioco della carta}
In questa fase di gioco chi ha dichiarato prima il tipo di contratto finale sarà la persona che giocherà.
Questa persona è definita il 'dichiarante', il compagno del dichiarante è definito il 'morto'.

\paragraph{Cosa fa il morto?}
Giocata la prima carta il morto depositerà le carte sul tavolo.
Da quel momento in poi il morto giocherà le carte stabilite dal dichiarante.

\paragraph{Chi deve giocare la prima carta?}
La prima carta da giocare spetta al giocatore a destra del morto.


I giocatori in senso orario giocheranno ciascuno una carta.
La carta più alta del seme giocato dal primo giocatore sarà quella che vincerà la mano.
Il giocatore che ha vinto la mano dovrà giocare per primo al turno seguente.

Una volta terminate le carte si può fare il conto del numero di prese fatte.

\subsection{Occorrente}

\begin{enumerate}
\item Servono 4 giocatori per formare 2 coppie.
In base alla modalità di torneo il numero di giocatori può aumentare.
Alcune varianti varianti permettono addirittura di giocare con un numero dispari.

\item Un mazzo di carte da 52 carte, escludendo i jolly.

\item Un bidding box per ogni giocatore.
Esso è una scatola contenente i cartellini che si possono usare per "parlare" durante il gioco.

\item Carta e penna per i risultati.
\end{enumerate}
\newpage
