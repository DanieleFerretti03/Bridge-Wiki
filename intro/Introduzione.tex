\section{Introduzione}

Questo documento è stato creato dalla mia personale esperienza.
Attualmente anche per il mio personale ed esclusivo uso.
Come ogni buon libro, non è completo, ma è volto ad evolvere con il tempo e con i consigli.
Sperando possa essere d'aiuto e ispirazione a chi lo legge e si affaccia al gioco.

Questo magnifico gioco è più bello se giocato con una buona compagnia di amici al proprio livello e non solo.


\subsection{Come leggere questo documento?}
Una domanda che mi sto facendo anche io mentre lo sto scrivendo.
Ci sono tre parti principali:

\begin{itemize}
\item Introduzione
\item Licita
\item Gioco della carta
\end{itemize}

Essendo i tre aspetti cardine di questo gioco, nello stesso capitolo potrete leggere di argomenti più avanzati altri più basilari.
Una volta letta l'introduzione, potete provare ad alternare un po' i capitoli seguenti.

Il mio suggerisco di non incaponirsi con il capire tutto alla prima lettura.
D'altronde, a breve vi accorgerete anche voi di quanto possa risultare complicato questo gioco.

\paragraph{Se leggete questo documento da completi novizi:}
Cercate di comprendere il senso generale del gioco.
Solo avendo la piena maestria delle basi potrete addentrarvi in argomenti piu' complessi.
Non ha nessun valore un giocatore esperto in una sola parte del gioco.
È bene avere una conoscenza a orizzontale, comprendo pian pianto tutti gli argomenti proposti.

\paragraph{Per giocatori un po' studiati:}
Cercate qualche argomenti che stuzzichi il vostro interesse.
Nel bridge ogni persona puo' dire cosa secondo essa e' la cosa migliore.
Cercare di capire se quello che sto scrivendo rispecchia il vostro (e della persona che gioca con voi) gusto.

\newpage
