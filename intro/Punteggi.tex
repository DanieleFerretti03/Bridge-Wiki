\section{Calcolo del punteggio}

\paragraph{A cosa serve?}
Affrontiamo questo argomento per due ragioni specifiche:
\begin{enumerate}
\item Saper misurare la qualità del proprio gioco a fine partita.
  Avere un punteggio significa avere la possibilità di confrontarsi con gli altri risultati.
  Se tutti hanno fatto 100 punti con le nostre carte e noi 130, noi siamo stati i migliori.
  Se tutti hanno fatto 600 punti, e noi solo 170, è evidente che ci siamo persi qualcosa.

\item Saranno presenti diverse situazioni di licita ove è bene saper effettuare questi conti in anticipo.
\end{enumerate}

\paragraph{Devo saperli prima di giocare?}
Oh no certo che no.
Avere un'idea minima dei punteggi aiuta, ma non è una capacità essenziale inizialmente.
In ogni caso, giocando con qualcuno di più esperto di voi sarà probabile che questo ricordi a memoria il punteggio dei vari contratti.
Vedrete che anche voi con il tempo inizierete a ricordare via via questi punteggi.
Vi sarà di aiuto alla memorizzazione scrivere come Sud.

\paragraph{Dove posso controllar questi punteggi?}
All'inizio il bidding box sarà vostro amico, potrete guardare dietro ogni contratto i relativi punteggi.
Li vi sono riportate anche tutte le possibili varianti.
Non abbiate timore di controllare voi stessi il punteggio, fidarsi è bene e non farlo è meglio.
Gli sbagli capitano anche ai migliori, e soprattutto dopo diverse mani giocate dove la stanchezza si fa sentire.

\paragraph{Cosa succede se andiamo sotto?}
I punti andranno segnati agli avversari.
Il quanto dipende dal numero di mani sotto al contratto stabilito.
Nota: Se contrato o surcontrato i punteggi variano seguendo quanto scritto nei cartellini.

\paragraph{Ma quindi è una cosa negativa andare sotto?}
Talune volte si, altre no.
Analizziamo i diversi scenari:

\begin{itemize}

\item Ipotizziamo che la coppia Est-Ovest dichiari e faccia un contratto di 6NT (990 punti se in prima).
  In questa situazione, se tutte le coppie Est-Ovest dovessero effettuare questo contratto, la coppia Nord-Sud non ne risentirebbe.
  La cosa migliore possibile in quel contratto era fagli fare 6NT.
  Il risultato del torneo in quella mano per la coppia Nord-Sud non verrà visto come una sconfitta.

\item Ipotizziamo che la coppia Nord-Sud dichiari 4\C! -4\footnote{Il punto esclamativo è una possibile notazione del contrato} (-800 punti se in prima).
  In questo frangente la copia Est-Ovest punisce un contratto eccessivo con un profitto notevole.
  Quei punti sono simili al piccolo slam citato prima, tuttavia Est-Ovest si sono limitati a giocare in difesa.
  Non c'è garanzia che avrebbero fatto più punti o gli stessi giocando in attacco.
  In questo frangente, andare sotto significa aiutare gli avversari.

\item Esistono infine della mani che verranno approfondire in seguito dette \textit{sacrifici}.
  Questo tipo di mano fanno segnare dei punteggi negativi, tuttavia scontano il vero punteggio della mano.
  Esempio: Giocare 5\C -1 (-50 se in prima) invece che far giocare 4\p agli avversari (-420 se in prima).
  In questo frangente tutte e due le mani fanno segnare punti agli avversari.
  Tuttavia, -50 è più piccolo di -420 cosa che ci fa preferire dargli solo -50.

  In questa situazione è come se il panettiere vi chiedesse se il pane lo volete pagare 2€ oppure 200€.
  Sfido qualcuno a dire che non preferirebbe pagarlo solo 2€.
  Lo stesso ragionamento viene applicato al punteggio, se siamo obbligati a far segnare un punteggio agli avversari, preferiamo dargli meno punti possibili.
\end{itemize}

\subsection{Il metodo generale}
Durante la licita non sarà possibile consultare il bidding box.
Illustriamo un procedimento per calcolare il punteggio di un certo contratto a mente.

\begin{enumerate}
  \item Conta il numero di punti apportati per ogni presa
  \item Aggiungi i bonus che spettano ad ogni livello licita
  \item Moltiplica per i possibili fattori supplementari
\end{enumerate}

\paragraph{Punti per ogni presa:}
Ad ogni livello di licita citato, il numero di prese da effettuare sarà di 6 + il livello citato.

\begin{center}
\begin{tabular} {| l l || l l|}
  \hline
  Contratto & Prese & Contratto & Prese \\
  \hline
   1 & 7 & 5 & 11\\
   2 & 8 & 6 & 12\\
   3 & 9 & 7 & 13\\
   4 & 10 & & \\
  \hline
\end{tabular}
Prese da fare dato il livello del contratto.
\end{center}

Dopo aver ottenuto questo numero, dovete moltiplicare per il punteggio illustrato in base al tipo di contratto:

\begin{center}
\begin{tabular} {| l c |}
  \hline
   Seme & Punteggio \\
   \hline
   Maggiore(\p, \C) & 30 \\
   Minore(\q, \f) & 20 \\
   Senza & 30, e dopo 20 \\
  \hline
\end{tabular}
Valore di ogni presa in relazione al seme.
\end{center}

\paragraph{In caso di prese in meno:}
In caso dichiarate un certo numero di prese, ma ne facciate di meno, parleremo di un contratto fallito.
In termini di punteggio non ci attribuiremo nessun punteggio, ma anzi, ne daremo agli avversari.
Quanti punti daremo agli avversari? Questo numero dipende da quante prese in meno faremo.
In prima 50 punti per ogni mano persa, 100 se in seconda.

\paragraph{Bonus per ogni livello}
L'unica prerogativa è quella di \textbf{dichiarare} il livello.
Se arrivassimo a fare tutte le prese, ma non le dichiarassimo, non ci spetterebbe il bonus dato dallo slam.
Questo gioco premia la precisione, non i colpi di fortuna.

Facciamo una breve digressione sul quali siano i vari tipi di licita raggiungibile:
\begin{itemize}
\item Parziale: se i punti ottenuti dalle prese sono inferiori ai 100.
\item Contratto: Se i punti ottenuti dalle prese sono maggiori di 100.
\item Piccolo slam:si intende i contratti del sesto livello.\footnote{Perdere solo una presa}
\item Slam: i contratti del settimo livello\footnote{Fare tutte le prese}.
\end{itemize}

La tabella seguente illustra i punteggi:

\begin{center}
\begin{tabular} {| l l l|}
  \hline
  Livello & Bonus in prima & Bonus in seconda\\
  \hline

  Parziale & 50 & 50\\
  Contratto & 300 & 500 \\
  Piccolo slam & 500 & 700\\
  Slam & 1000 & 1500\\
  \hline
\end{tabular}
\end{center}


\subsection{Corollari ai punteggi}
Quando si dichiara bisogna tenere a mente se aumentare di livello è impattante a livello di punteggio.
Non sempre è conveniente impegnarsi a fare più prese.

\begin{itemize}

\item Es. Di quando conviene: 2\p+2 e 4\p.
Il conto risulta essere per 2\p+2: $4*30 + 50 = 170$, mentre per 4\p: $4*30 + 300 = 420$.

\item Es. Di quando non conviene:2\q+1 e 3\q.
Il conto brevemente: 2\q+1 $= 20*3 +50$; mentre 3\q $= 20*3 + 50$

\end{itemize}
Se crediamo di poter fare 10 prese con un contrato a \p, non possiamo fermarci a 2\p.
In quel frangente sarà importante dichiararne 4\p, poiché la differenza in punteggio è sensibile.
Viceversa, se durante la licita siamo arrivati a 2\q e pensiamo di arrivare a 9 prese, la cosa più conveniente è passare.
Alzando il livello obblighiamo a dover fare una presa in più senza nessun beneficio di punteggio.

\begin{center}
\begin{tikzpicture}
\begin{axis}[
  xmin = 0, ymin = 0, xmax = 7, ymax = 1520,
  xlabel = contratto, ylabel = punteggio,
  ytick={0,100,200,300, 400, 500, 600, 700, 800, 900, 1000, 1100, 1200, 1300, 1400, 1500},
  ]

  \addplot[color = black] coordinates{
    (0,0)(1, 70)(2, 90)(3, 110)(4, 130)(5, 400)(6, 920)(7, 1440)
  }
  node[rotate = 45, below, pos = 0.5]{\club \diamond};

  \addplot[color = red] coordinates{
    (0,0)(1, 80)(2, 110)(3, 140)(4, 420)(5, 450)(6, 980)(7, 1510)
  }
  node[rotate = 45, above, pos = 0.2]{\heart \spade}; 

  \addplot[color = green] coordinates{
    (0,0)(1, 90)(2, 120)(3, 400)(4, 430)(5, 460)(6, 990)(7, 1520)
  }
  node[rotate = 65, above, pos = 0.7]{Senza}; 
\end{axis}
\end{tikzpicture}
\end{center}
