\section{La giusta persona con cui giocare}

Non è facile trovare una persona in sintonia con noi e i nostri modi di fare.

\paragraph{Livello di gioco:}
Servirà trovare una persona con più o meno il vostro livello di gioco.
Se troverete una persona con molta più bravura rischierete di essere d'intralcio con la vostra inesperienza.
Viceversa, una persona molto più scarsa di voi potrebbe trascinarvi al suo livello di gioco.
Con conseguente rischio di non migliorare mai.

\paragraph{Conoscenze del gioco:}
È importante parlare la stessa lingua.
Questo gioco è basato sulla comunicazione dei cartellini.
Senza avere la complicità comune sul significato delle dichiarazioni non potrete andare da nessuna parte.
Se 2\p per il vostro compagno è debole e per voi forte avrete l'impiccio di non arrivare mai al giusto contratto.

\paragraph{Spirito competitivo:}
È importante avere un livello di competività simile.
Chi è molto competitivo vuole discutere delle possibili variabili della partita appena giocata.
Chi lo è poco pensa alla prossima mano e dimentica quella appena giocata.
Fidatevi, non è bello giocare con una persona che brontola per ogni vostra giocata.

\paragraph{Cambiare compagnia:}
È facile che vi ritroverete a giocare con persone diverse.
Una persona potrebbe giocare il martedì e non il giovedì, e quindi voi cambiare persona con cui fare coppia.
In queste situazioni matematicamente certo che i due avranno modi di giocare diversi.
Prendetevi un momento per ripassare quali convenzioni adottate con la data persona.

\paragraph{Litigare è male:}
Alle volte la persona davanti a voi farà delle cose che voi reputate sbagliate.
Abbiate garbo nel segnalare l'errore e cercate di non puntualizzare tutto.
Non sembra, ma anche l'altra è una persona come voi.
Sentirsi riprendere non fa piacere a nessuno.
Cerate di rendere i momenti di condivisione costruttivi a migliorare.
Evitate in tutti i modi di renderlo un momento di litigio.
