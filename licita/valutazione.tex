\section{La valutazione della mano}

Distribuite le carte, la prima cosa che vogliamo capire è: quanto è forte la nostra mano?

\paragraph{Sistema 4-3-2-1:}
Questo modo di valutare la mano è stato pubblicato nel 1915, e fa da base per quelli moderni.
+4 Asso, +3 Re, +2 Regina, +1 Fante, +0 per tutte le altre carte.
Questo sistema consente di stabilire quanti \textit{punti onore} possediamo.

\paragraph{Perché sono nati altri sistemi?}
Con l'aumentare della competizione, ci si è accorti che in diverse situazione il sistema 4-3-2-1 non da una reale valutazione della mano.
Chi gioca-giocava si è presto accorto che rispondere alla domanda "quanto è forte la mia mano", non è affatto una questione semplice.
Vediamo alcuni concetti sul perché questa operazione risulta difficoltosa:
\begin{itemize}
\item La mano, rimanendo la stessa per tutta la licita, viene rivalutata di volta in volta.
  Avere una conoscenza delle altre mani può aumentare la nostra forza o diminuirla.

\item Valutare alcune mani particolari richiede una buona conoscenza del gioco della carta.
  Certe combinazioni potrebbero essere favorevoli per delle manovre del gioco.

\item Avere due mani con lo stesso numero di punti non significa che siano da trattare allo stesso modo.
  Alcune di esse potrebbero essere migliori in difesa rispetto che in attacco.
  O logicamente, vale anche il viceversa.

\item Ci sono molti fattori che possono influenzare le considerazioni che facciamo sulla nostra mano.

\item Alcuni considerazione potrebbero portare a rimuovere dei punti.
  Vedi ad esempio un Re secco.\footnote{Con secco si intende senza altre carte in quel seme.} 

\item Non contano solo le carte possedute, ma anche come sono disposte.
  Avere molte carte in un seme diventa un vantaggio.
  Diversamente, una mano molto equilibrata nella distribuzione non genera nessuna plusvalenza.
\end{itemize}
Compreso che non è una questione facile, invito a ripassare in questa sezione più volte.

\paragraph{Un piccolo trucco:}
Per contare velocemente si può raggruppare il gruppo di carte dall'asso al fante.
Questo gruppo di carte fa 10 punti.

\paragraph{Il seme ci interessa?}
In linea di massima avere un asso di \f o \C non cambia nulla.
Un asso vale sempre come un asso, ovvero +4.

Nota: Degli onori in un palo del nostro compagno possono acquisire un valore maggiore.
Mentre, viceversa, un onore in un seme dove gli avversari vogliono giocare, potrebbe ricevere della svalutazione.

\paragraph{La distribuzione: }
Nella valutazione della mano avere tante carte in un solo seme è favorevole.
Esiste una mano detta \textit{la bara del bridge} fatta così: 4-3-3-3.

Alcuni giocatori si attribuiscono dei punti in più guardando i pali lunghi.
Altri invece decidono di guardare i pali corti.

Il secondo è quello più comune, ed è fatto nel seguente modo:
+3 per un vuoto, +2 per un singolo, +1 per un doubleton.

\paragraph{I tipi di mano:}

Non è assolutamente importate sapere i nomi per saper giocare.
È comodo avere questa distinzione per essere meno verbosi. 
\begin{itemize}
\item Mano bianca: una mano con nessun punto o quasi.
\item Apertura minima: dai 11 ai 14 P.O.
\item Apertura: dai 15 ai 17 P.O.
\item Mano di rovescio: 18-20 P.O.
\item Mano forte: 20+ P.O.
\end{itemize}

\paragraph{Il concetto di onori scoperti:}
La distribuzione ci aiuta regalando dei punti, ma talvolta può anche fare dei danni.
Es. Il re singolo, una regina seconda, o un fante terzo diventano inutilizzabili.
Perché? Il re singolo viene perso quando gli avversari incassano l'asso.
E lo stesso ragionamento si applica alle altre combinazioni citate.

Per queste situazioni, è bene considerare poco o nulla quegli onori nel calcolo del punteggio.
Perché non nulla e basta? Potrebbero succede delle situazione dove gli onori scoperti vengono comunque incassati.
