\section{Mani forti}

Nella dichiarazione del primo di mano si potrebbero avere 20 o più punti.
In questo frangente abbiamo quasi la manche assicurata.
Tuttavia, sarà il caso far sapere fin da subito all'altra persona la portata della nostra mano.

\paragraph{Cosa dobbiamo licitare?}
In questo caso, si apre la mano con 2\f, che nota non è un tentativo di sbarramento.
Immaginate che questo cartellino significhi "Ho una mano particolarmente forte".
La forza potrebbe essere data dai punti o anche da una distribuzione molto sbilanciata.

Una volta che il nostro compagno ci avrà risposto, "descrivi la tua mano".
Avendo un colore sbilanciato comunicheremo quello, viceversa, avendo una mano bilanciata da 22 punti diremo 2NT.

\paragraph{Che cartellino è "descrivi la tua mano"?}
Come ogni convenzione del bridge, esistono diverse varianti.
Illustro di seguito la più semplice:
si risponde sempre 2\q di attesa.
Se gli avversari intervengono si passa.

\paragraph{Che altre convenzioni esistono?}
Nota, sconsiglio l'uso.
Esistono delle convenzioni come la Crodo che comunicano fin da subito all'apertore il numero di assi.

Perché la sconsiglio?
\begin{itemize}
\item Rischia di far diventare la mano forte quella del morto e non del dichiarante.
\item Inverte il capitanato inutilmente, l'apertore ha già descritto praticamente tutto della sua mano.
\item De delle informazioni alla difesa.
\item Ammettiamo che il rispondente comunichi di avere un buon numero di punti.
  Allora significa che gli avversari non potrebbero mai fare ostruzioni di nessun tipo.
  Quindi, potreste arrivare a comunicare le stesse informazioni in altro modo.
\item Può dar nascita a dei contro convenzionali per segnalare attacchi più facilmente.
\end{itemize}

\paragraph{Il lato negativo di 2\q}
Se il colore dove siamo sbilanciati fosse \q, allora il nostro compagno ci anticiperebbe il colore.
Facendo giocare alla mano debole il contratto.
Ecco perché si preferisce con quel tipo di mano usare la convenzione 2\q multy.

\paragraph{Esempi pratici:}
Vediamo un esempio dove non si arriva al contratto:
\begin{center}
\begin{auction}
  {}, {2C}, {p}, {2D}, 
  {p}, {2S}, {p}, {p},
  {p}
\end{auction}
\end{center}

\begin{itemize}
\item Nord comunica con 2\f "Ho una mano forte".
\item Sud con 2\q "Descrivi la tua mano"
\item Nord con 2\p "Ho almeno 5 carte di \p".
\item Sud passando fa capire al compagno che non c'è possibilità di fare più di un parziale.
  Potrebbe avere una mano da 0 punti ad esempio.
\end{itemize}

Vediamone uno in cui si arriva a contratto:
\begin{center}
\begin{auction}
  {}, {2C}, {p}, {2D}, 
  {p}, {3C}, {p}, {5C},
  {p}
\end{auction}
\end{center}

\begin{itemize}
\item Nord comunica con 2\f "Ho una mano forte".
\item Sud con 2\q "Descrivi la tua mano"
\item Nord con 3\f "Ho almeno 5 carte di \f".
\item Sud dicendo 5\f comunica appoggio a fiori e punti sufficienti per arrivare a contratto.
\end{itemize}

Vediamo un esempio di dichiarazione per lo slam:

\begin{center}
\begin{auction}
  {}, {2C}, {p}, {2D}, 
  {p}, {2H}, {p}, {3H},
  {p}, {4NT}, {p}, {6H},
  {p}, {p}, {p}
\end{auction}
\end{center}

\begin{itemize}
\item Nord comunica con 2\f "Ho una mano forte".
\item Sud con 2\q "Descrivi la tua mano"
\item Nord con 2\C "Ho almeno 5 carte di \C".
\item Sud dicendo 3\C comunica un interesse per lo slam.
  Se avesse pochi punti passerebbe, tra 2\C+1 e 3\C il punteggio è lo stesso.
  Se avesse solo il punteggio da contratto direbbe direttamente 4\C.
  Dicendo 3\C sta facendo una dichiarazione che indica appoggio, ma di continuare a parlare.
\item Nord in questo esempio chiede direttamente gli assi e Sud manda al piccolo slam.
\end{itemize}


Vediamo un esempio di mano bilanciata: 

\begin{center}
\begin{auction}
  {}, {2C}, {p}, {2D}, 
  {p}, {2NT}, {p}, {3NT},
  {p}, {p}, {p}
\end{auction}
\end{center}

\begin{itemize}
\item Nord comunica con 2\f "Ho una mano forte".
\item Sud con 2\q "Descrivi la tua mano"
\item Nord con 2 NT "Ho una mano bilanciata da 22 punti"
\item Sud decide di chiudere a 3NT per arrivare a contratto. (Probabilite avrà 3-5 punti i mano o poco più)
\end{itemize}


