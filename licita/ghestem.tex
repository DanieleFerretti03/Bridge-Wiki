\section{Interventi in bicolore}

\paragraph{Di cosa stiamo parlando?}
Iniziamo con il definire cosa intendiamo con bicolore.
Una mano bicolore e' una mano con preponderanza in due semi.

Nello specifico, almeno 5 carte in entrambi i semi.
Vediamo degli esempi:

\begin{enumerate}
\item \hand{xxxxx,xxxxx,xx,x}
\item \hand{xxx,xxxxx,xxxxx,}
\item \hand{xx,xxxxx,x,xxxxx}
\end{enumerate}

In questo caso specifico, parliamo di intervento in bicolore.
Ovvero, se siamo la successiva persona a dover licitare dopo
una dichiarazione avversaria.
Vediamo una situazione pratica:

\hand{x, xx, AKxxx, AKxxx}

\begin{center}
\begin{auction}
  {1S}, {?}
\end{auction}
\end{center}

Avendo la seguente mano cosa licitiamo?
Qualcuno potrebbe suggerire di dire:
\begin{itemize}
  \item Contro
  \item 2\f
  \item 2\q
  \item passo
\end{itemize}

Vediamoli uno per uno e capiamo perche' ci sono delle ciriticita'.
Il primo comunica esclusivamente che si possiede un'apertura.
In base agli accordi presi pero' questa potrebbe essere un licita poco corretta.
Il contro sul maggiore ci farebbe intendere di avere in questo caso almeno 4\C.
In ogni caso, non dice nulla ne delle fiori, ne delle quadri.

Gli interventi di 2\f e 2\q sono piu' descrittivi ma omettono comunque
l'altro palo.
E siccome la sfortuna alle volte puo' essere nera, magari abbiamo proprio
un incontro sul palo non licitato e nessuno in quello licitato.
Intendiamoci, magari invece potrebbe esserci l'incontro nel palo licitato.
Faccio pero' notare che non e' detto questo incontro.

Il passo e' messo li per varie ragioni.
Qualcuno con quella mano potrebbe decidere o essere obbligato da sistema, a passare.
Notiamo che questo tra tutti e' proprio l'intervento peggiore. 
Perche'? Eh bhe, non compete e lascia la strada libera alla licita.

\paragraph{Facciamo due conti}
Qualcuno potrebbe pensare che questa sezione tratta
di una situazione molto poco probabile.
Quando mai si vede quel tipo di mano?

Ebbene, facendo due conti applicando la formula di Bayes.
Usiamo le seguenti convenzioni:
\begin{enumerate}
\item $C$: Evento di trovarci nella situazione di dover usare questa convenzione.
\item $PO_a$: Che il nostro avversario abbia almeno 12 P.O.
\item $PO_m$: Che nella nostra mano ci siano almeno 8 P.O.
\item $Dis$: Che nella nostra mano ci sia una distribuzione almeno 5-5.
\end{enumerate}

$$P(C) = P(Dis) * \frac{P(PO_a | PO_m) * P(PO_m)}{P(PO_a)}$$

Facendo un po' di sostituzioni che si possono verificare, otteniamo:
$$P(C) = 21.98\% * 39.5\% = 8.68\% $$

Ovvero, giocando regolarmente dovrebbe capitare
\textbf{approssimativamente} una volta ogni 11.5 mani.

\paragraph{Dovrei usarla?}
Per quello che sono i conti di prima,
avere nel proprio arsenale questa convenzione puo' aiutare molto.
Dopodiche', esistono altri sistemi che abbiamo discusso prima
che potrebbero funzionare a trovare un incontro.
Certo e' che non e' una convenzione "semplice", quindi suggerisco
di giocarla con una persona dopo un po' di tempo e non a primo acchito.

\paragraph{Come funziona la convenzione nel pratico?}

Riporto di seguito la tabella delle varie combinazioni.
Provo dopo a spiegare con che metodo io ho memorizzato questa tabella.

\begin{tabular}{|l|lll|}
\hline
Apertura & Surlicita & 2NT   & 3\f   \\
\hline
1\f      & \C+\p     & \q+\C & \q+\p \\
1\q      & \C+\p     & \f+\C & \f+\p  \\
1\C      & \f+\p     & \f+\q & \q+\p \\
1\p      & \f+\C     & \f+\q & \q+\C \\
\hline
\end{tabular}

Nota: La surlicita 1\f-2\f indica una lunga di fiori.
Pertanto, dove e' indicato surlicita nella licita di 1\f si usa comunque 2\q.

\paragraph{Aiuto per la memorizzazione}

\begin{itemize}
  \item 2NT: Il pattern dentro questa licita e' che indica sempre i due semi piu' bassi.

  \item Surlicita nei minori: Indica sempre i due semi maggiori.
  \item Surlicita nei maggiori: Avendo escluso la coppia dei due minori.
  Ci rimangono le coppie di minore e maggiore.
  In questo caso, per evitare di dire noi le fiori la surlicita indica la bicolore con fiori e l'altro maggiore.
  \item 3\f nei maggiori: per il ragionamento di prima, evitando di dire noi le quadri, indichiamo le quadri e l'altro maggiore.

  \item 3\f nei minori: Escludendo i due piu' bassi e i due piu' alti come coppia, ci rimane solo una coppia.
  Il caso sfortunato vuole che in una di queste situazione anticipiamo il palo delle fiori.
\end{itemize}
