\section{Aperture di sbarramento}

Solitamente gioca la linea che ha più punti.
Esiste tuttavia una situazione in cui anche con meno punti siamo interessati a giocare.

\paragraph{Competere con la distribuzione:}
Avendo un seme molto lungo, è probabile che saremo in grado di stabilire tante prese se quello diventasse l'atout.
Certo varrebbe ben poco se non lo fosse.
Ecco che quindi, un giocatore con un palo di 6+ carte potrebbe valutare di competere.

\begin{center}
\begin{auction}
  {}, {3S}, {p}, {p}, 
  {p}
\end{auction}
\end{center}


\paragraph{Le coppie competitive:}
Non è sempre vero che si gioca avendo più punti.
Ci sono delle eccezioni con delle coppie competitive che riescono a dichiarare anche con meno punti sulla linea.
Può inoltre succedere che non ci sia un incontro e che di conseguenza non si riesca a giocare.

In generale tuttavia è una "massima" corretta.

\paragraph{Attenzione ai punti}
Abbiamo detto che può succedere di giocare con meno punti sulla linea.
Serve fare attenzione tuttavia a quanti punti in meno abbiamo!

Prendiamo ed esempio questo tipo di mano:

\begin{center}
\begin{auction}
  {}, {1S}, {4H}, {p}, 
  {p}, {p}
\end{auction}
\end{center}

\begin{center}
\hand{x,xxxxxxxx,xx,xx}
\end{center}
Avendo questa lunga di cuori, un baragge possibile è 4\C ?
Se vogliamo competere esclusivamente guardando la distribuzione,
la risposta intuitiva sarebbe si.

Campiamo però perché con questo tipo di mano è sconsigliabile dichiarare 4\C.
Prendiamo in esame il caso peggiore di tutti, la nostra mano ha 0 punti
e anche quella della persona che gioca con noi.
Conseguenza logica della ipotesi? La linea avversaria ha 40 punti!

In questa situazione, gli avversari con tutta probabilità ci cotreranno.
La nostra mano vede 5 perdenti (1\p, 2\q, 2\f), ipotizziamo
che nella mano del morto non troviamo una distribuzione favorevole e che quindi le 5
nei semi non siano in nessun modo scartabili.

In questa situazione siamo già sotto di una presa e contrati!
Ma non abbiamo ancora considerato il palo di \C dove non abbiamo
neanche un onore.
Noi contiamo 8 carte, e mancanti sono 5 da dividere tra 3 giocatori.
Caso peggiore di tutti? 5 carte tutte da una parte per la difesa.

Se fosse così saremo sotto di altre 5 prese per un totale di 10 sotto contrati!
Ricordo che: 4\C - 10! in prima -2600 e in seconda sono: -2900.

\paragraph{La difesa dal caso peggiore}
Qualcuno potrebbe obbiettare che nel punto precedente sono stato troppo catastrofico.
Osservazione corretta, tutte le sfortune in una volta sola sono difficili che accadano.

Tuttavia, anche senza considerare il caso peggiore: (4\C -2! sono -500 se in seconda).
Per evitare di fare del barrage vantaggioso quindi ci diamo una "regola".

\paragraph{Regole per fare barrage}
Sono poche e la loro ragione è intuibile dagli esempi portati prima:
\begin{enumerate}
  \item Vogliamo almeno 5+
  \item Non vogliamo avere punteggio d'apertura 12-
    Se avessimo l'apertura possiamo competere normalmente.
  \item Vogliamo avere almeno 2 onori nel palo in cui facciamo barrage.
\end{enumerate}
