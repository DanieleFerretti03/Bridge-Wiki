\section{Apertura del primo di mano}

La licita inizia dal giocatore che ha dato le carte.
Oppure quello con la "D" sopra il board.

\paragraph{Parare o non parlare?}
Questo giocatore in prima battuta deve effettuare la valutazione della mano.
Conoscendo il punteggio onore associato sarà possibile capire se è possibile dichiarare.

\paragraph{Quanto punti al minimo servono?}
Una mano per essere aperta in prima posizione ha bisogno di almeno 12 punti onore.
Alcuni sistemi diversi, o giocatori più aggressivi potrebbero non rispettare questa regola.
Nota, fanno eccezione dichiarazioni di tipo preventivo o convenzionale.

\paragraph{La dichiarazione più semplice di tutte}
Il cartellino verde, con la scritta passo è quella più semplice di tutte.
Non hai motivo di parlare? Passi. Non hai sufficienti punti per parlare? Anche in questo caso passi.

Esistono delle situazioni convenzionali in cui questo passo può diventare ricco di informazioni.
Tuttavia, questo cartellino in generale indica: Non ho interesse a giocare.

\paragraph{Allenamento sul passare o non passare}
Di seguito ci saranno diverse mani, stabilisci se queste mani vanno o meno passate.
Ti ricordo, per il momento conta avere 12 punti onore.
Se vedi una 'x', significa una carta qualsiasi dal 2 al 9.

\begin{table}[]
\begin{tabular}{|cl|l|l|}
\hline
Numero & Mano & Tentativo & Soluzione \\
\hline
1.&\hand{Axxx,Kxx,Qxx,Jxx}& & Passo\\
\hline
2.&\hand{Ax,KJxxx,Qxx,Jxx}& & Passo\\
\hline
3.&\hand{AQ,KJxxx,Qxx,Jxx}& & Dichiarare\\
\hline
4.&\hand{AQ,KJx,AKQxx,Jxx}& & Dichiarare\\
\hline
5.&\hand{xx,KJx,Axxxx,Jxx}& & Passo\\
\hline
6.&\hand{xxxxx,,Axxxx,Jxx}& & Passo\\

\hline
\end{tabular}
\end{table}

