\section{Apertura mani bilanciate}
Una mano può essere di due tipi, sbilanciata o bilanciata.
Più una mano è bilanciata più ci interesserà giocarla senza un atout.

\paragraph{Cosa significa avere una mano bilanciata?}
Significa avere carte in tutti i pali, e avere al più un palo con due carte.

\paragraph{Tutte le volte che ho una mano bilanciata cose devo fare?}
Inizia sempre con il contare i punti!
Se non hai i famosi 12 punti, non andiamo da nessuna parte.

Ecco gli scaglioni di punteggio e il relativo comportamento (Nota, questi sono quelli adottata da me, altre coppie hanno altri modi di comunicare):
\begin{itemize}
\item -12: Non è da aprire
\item 12-14: Aprire del palo e poi dire 1NT se non c'è stato un appoggio.
\item 15-17: Aprire di 1NT
\item 18-19: Aprire di un palo e poi licitare per mostrare mano di rovescio con 2NT.
\item 20-21: Vedi 2\q multy.
\item 22: Aprire di 2NT
\item 22+: Vedi 2\f forte.
\end{itemize}

\paragraph{Apertura 1NT debole:}
Nota bene, sconsiglio di giocarla.
Questa convenzione fa eccezione rispetto all'avere un'apertura in mano.
Se da una parte aprire di 1NT con 10-12 punti genera sbarramento, dall'altra è facile che venga contrato per punire.

\paragraph{Mano limitata:}
Se avete da 12-14 punti, dopo aver licitato un palo, non avete la forza per dirne un secondo.
Ricorrete dunque a dire 1NT.

\begin{center}
\begin{auction}
  {}, {1D}, {p}, {1S}, 
  {p}, {1NT}, {p}, {p},
  {p}
\end{auction}
\end{center}

Una possibile mano di Nord:
\begin{center}
\hand{AK,QJxx,Qxxx,Kxx}
\end{center}

\paragraph{Apertura 1NT forte:}
Esistono di questa convenzione diverse varianti di punteggio.
La variante moderna è quella di avere dai 15-17 punti.
Logicamente la mano è bilanciata.

\paragraph{La distribuzione particolare:}
Una mano con questa distribuzione, per la definizione che ci siano dati, è bilanciata: 5-3-3-2
Tuttavia, alcune coppie, possedendo 5 carte in un palo nobile preferiscono non aprire di 1NT avendo il punteggio.
Il mio consiglio è di farlo, sarete più precisi con il punteggio in una sola dichiarazione.

Detto questo, discutetene in modo tale da potervi adattare di conseguenza.
Poiché, aprire con questa distribuzione necessita della "puppet Stayman" e non della Stayman classica.
