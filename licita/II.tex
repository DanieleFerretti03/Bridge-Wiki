
\begin{tikzpicture}[remember picture,overlay]
  \node[align=center] at (current page.center) {
    {\fontsize{60pt}{72pt}\selectfont \textbf{II Parte}}
  };
\end{tikzpicture}

\newpage

\def\II{licita}

\section{La valutazione della mano}

Distribuite le carte, la prima cosa che vogliamo capire è: quanto è forte la nostra mano?

\paragraph{Sistema 4-3-2-1:}
Questo modo di valutare la mano è stato pubblicato nel 1915, e fa da base per quelli moderni.
+4 Asso, +3 Re, +2 Regina, +1 Fante, +0 per tutte le altre carte.
Questo sistema consente di stabilire quanti \textit{punti onore} possediamo.

\paragraph{Perché sono nati altri sistemi?}
Con l'aumentare della competizione, ci si è accorti che in diverse situazione il sistema 4-3-2-1 non da una reale valutazione della mano.
Chi gioca-giocava si è presto accorto che rispondere alla domanda "quanto è forte la mia mano", non è affatto una questione semplice.
Vediamo alcuni concetti sul perché questa operazione risulta difficoltosa:
\begin{itemize}
\item La mano, rimanendo la stessa per tutta la licita, viene rivalutata di volta in volta.
  Avere una conoscenza delle altre mani può aumentare la nostra forza o diminuirla.

\item Valutare alcune mani particolari richiede una buona conoscenza del gioco della carta.
  Certe combinazioni potrebbero essere favorevoli per delle manovre del gioco.

\item Avere due mani con lo stesso numero di punti non significa che siano da trattare allo stesso modo.
  Alcune di esse potrebbero essere migliori in difesa rispetto che in attacco.
  O logicamente, vale anche il viceversa.

\item Ci sono molti fattori che possono influenzare le considerazioni che facciamo sulla nostra mano.

\item Alcuni considerazione potrebbero portare a rimuovere dei punti.
  Vedi ad esempio un Re secco.\footnote{Con secco si intende senza altre carte in quel seme.} 

\item Non contano solo le carte possedute, ma anche come sono disposte.
  Avere molte carte in un seme diventa un vantaggio.
  Diversamente, una mano molto equilibrata nella distribuzione non genera nessuna plusvalenza.
\end{itemize}
Compreso che non è una questione facile, invito a ripassare in questa sezione più volte.

\paragraph{Un piccolo trucco:}
Per contare velocemente si può raggruppare il gruppo di carte dall'asso al fante.
Questo gruppo di carte fa 10 punti.

\paragraph{Il seme ci interessa?}
In linea di massima avere un asso di \f o \C non cambia nulla.
Un asso vale sempre come un asso, ovvero +4.

Nota: Degli onori in un palo del nostro compagno possono acquisire un valore maggiore.
Mentre, viceversa, un onore in un seme dove gli avversari vogliono giocare, potrebbe ricevere della svalutazione.

\paragraph{La distribuzione: }
Nella valutazione della mano avere tante carte in un solo seme è favorevole.
Esiste una mano detta \textit{la bara del bridge} fatta così: 4-3-3-3.

Alcuni giocatori si attribuiscono dei punti in più guardando i pali lunghi.
Altri invece decidono di guardare i pali corti.

Il secondo è quello più comune, ed è fatto nel seguente modo:
+3 per un vuoto, +2 per un singolo, +1 per un doubleton.

\paragraph{I tipi di mano:}

Non è assolutamente importate sapere i nomi per saper giocare.
È comodo avere questa distinzione per essere meno verbosi. 
\begin{itemize}
\item Mano bianca: una mano con nessun punto o quasi.
\item Apertura minima: dai 11 ai 14 P.O.
\item Apertura: dai 15 ai 17 P.O.
\item Mano di rovescio: 18-20 P.O.
\item Mano forte: 20+ P.O.
\end{itemize}

\paragraph{Il concetto di onori scoperti:}
La distribuzione ci aiuta regalando dei punti, ma talvolta può anche fare dei danni.
Es. Il re singolo, una regina seconda, o un fante terzo diventano inutilizzabili.
Perché? Il re singolo viene perso quando gli avversari incassano l'asso.
E lo stesso ragionamento si applica alle altre combinazioni citate.

Per queste situazioni, è bene considerare poco o nulla quegli onori nel calcolo del punteggio.
Perché non nulla e basta? Potrebbero succede delle situazione dove gli onori scoperti vengono comunque incassati.

\section{Apertura del primo di mano}

La licita inizia dal giocatore che ha dato le carte.
Oppure quello con la "D" sopra il board.

\paragraph{Parare o non parlare?}
Questo giocatore in prima battuta deve effettuare la valutazione della mano.
Conoscendo il punteggio onore associato sarà possibile capire se è possibile dichiarare.

\paragraph{Quanto punti al minimo servono?}
Una mano per essere aperta in prima posizione ha bisogno di almeno 12 punti onore.
Alcuni sistemi diversi, o giocatori più aggressivi potrebbero non rispettare questa regola.
Nota, fanno eccezione dichiarazioni di tipo preventivo o convenzionale.

\paragraph{La dichiarazione più semplice di tutte}
Il cartellino verde, con la scritta passo è quella più semplice di tutte.
Non hai motivo di parlare? Passi. Non hai sufficienti punti per parlare? Anche in questo caso passi.

Esistono delle situazioni convenzionali in cui questo passo può diventare ricco di informazioni.
Tuttavia, questo cartellino in generale indica: Non ho interesse a giocare.

\paragraph{Allenamento sul passare o non passare}
Di seguito ci saranno diverse mani, stabilisci se queste mani vanno o meno passate.
Ti ricordo, per il momento conta avere 12 punti onore.
Se vedi una 'x', significa una carta qualsiasi dal 2 al 9.

\begin{table}[]
\begin{tabular}{|cl|l|l|}
\hline
Numero & Mano & Tentativo & Soluzione \\
\hline
1.&\hand{Axxx,Kxx,Qxx,Jxx}& & Passo\\
\hline
2.&\hand{Ax,KJxxx,Qxx,Jxx}& & Passo\\
\hline
3.&\hand{AQ,KJxxx,Qxx,Jxx}& & Dichiarare\\
\hline
4.&\hand{AQ,KJx,AKQxx,Jxx}& & Dichiarare\\
\hline
5.&\hand{xx,KJx,Axxxx,Jxx}& & Passo\\
\hline
6.&\hand{xxxxx,,Axxxx,Jxx}& & Passo\\

\hline
\end{tabular}
\end{table}


\section{Aperture II}

Se non sei il dealer, e questo passa, ti potresti trovare nella posizione di aprire per primo.
Chi apre stabilisce che in quella mano c'è sufficiente motivo per competere.
Diverso è giocare quando si ha solo un giocatore che ha passato, due o tre.

Valgono le regole citate per il primo in apertura.


% Apertura del secondo, terzo e quarto di mano.
% Regole di apertura speciali, del 20 e numero di picche.

\section{Apertura mani bilanciate}
Una mano può essere di due tipi, sbilanciata o bilanciata.
Più una mano è bilanciata più ci interesserà giocarla senza un atout.

\paragraph{Cosa significa avere una mano bilanciata?}
Significa avere carte in tutti i pali, e avere al più un palo con due carte.

\paragraph{Tutte le volte che ho una mano bilanciata cose devo fare?}
Inizia sempre con il contare i punti!
Se non hai i famosi 12 punti, non andiamo da nessuna parte.

Ecco gli scaglioni di punteggio e il relativo comportamento (Nota, questi sono quelli adottata da me, altre coppie hanno altri modi di comunicare):
\begin{itemize}
\item -12: Non è da aprire
\item 12-14: Aprire del palo e poi dire 1NT se non c'è stato un appoggio.
\item 15-17: Aprire di 1NT
\item 18-19: Aprire di un palo e poi licitare per mostrare mano di rovescio con 2NT.
\item 20-21: Vedi 2\q multy.
\item 22: Aprire di 2NT
\item 22+: Vedi 2\f forte.
\end{itemize}

\paragraph{Apertura 1NT debole:}
Nota bene, sconsiglio di giocarla.
Questa convenzione fa eccezione rispetto all'avere un'apertura in mano.
Se da una parte aprire di 1NT con 10-12 punti genera sbarramento, dall'altra è facile che venga contrato per punire.

\paragraph{Mano limitata:}
Se avete da 12-14 punti, dopo aver licitato un palo, non avete la forza per dirne un secondo.
Ricorrete dunque a dire 1NT.

\begin{center}
\begin{auction}
  {}, {1D}, {p}, {1S}, 
  {p}, {1NT}, {p}, {p},
  {p}
\end{auction}
\end{center}

Una possibile mano di Nord:
\begin{center}
\hand{AK,QJxx,Qxxx,Kxx}
\end{center}

\paragraph{Apertura 1NT forte:}
Esistono di questa convenzione diverse varianti di punteggio.
La variante moderna è quella di avere dai 15-17 punti.
Logicamente la mano è bilanciata.

\paragraph{La distribuzione particolare:}
Una mano con questa distribuzione, per la definizione che ci siano dati, è bilanciata: 5-3-3-2
Tuttavia, alcune coppie, possedendo 5 carte in un palo nobile preferiscono non aprire di 1NT avendo il punteggio.
Il mio consiglio è di farlo, sarete più precisi con il punteggio in una sola dichiarazione.

Detto questo, discutetene in modo tale da potervi adattare di conseguenza.
Poiché, aprire con questa distribuzione necessita della "puppet Stayman" e non della Stayman classica.


\section{Mani forti}

Nella dichiarazione del primo di mano si potrebbero avere 20 o più punti.
In questo frangente abbiamo quasi la manche assicurata.
Tuttavia, sarà il caso far sapere fin da subito all'altra persona la portata della nostra mano.

\paragraph{Cosa dobbiamo licitare?}
In questo caso, si apre la mano con 2\f, che nota non è un tentativo di sbarramento.
Immaginate che questo cartellino significhi "Ho una mano particolarmente forte".
La forza potrebbe essere data dai punti o anche da una distribuzione molto sbilanciata.

Una volta che il nostro compagno ci avrà risposto, "descrivi la tua mano".
Avendo un colore sbilanciato comunicheremo quello, viceversa, avendo una mano bilanciata da 22 punti diremo 2NT.

\paragraph{Che cartellino è "descrivi la tua mano"?}
Come ogni convenzione del bridge, esistono diverse varianti.
Illustro di seguito la più semplice:
si risponde sempre 2\q di attesa.
Se gli avversari intervengono si passa.

\paragraph{Che altre convenzioni esistono?}
Nota, sconsiglio l'uso.
Esistono delle convenzioni come la Crodo che comunicano fin da subito all'apertore il numero di assi.

Perché la sconsiglio?
\begin{itemize}
\item Rischia di far diventare la mano forte quella del morto e non del dichiarante.
\item Inverte il capitanato inutilmente, l'apertore ha già descritto praticamente tutto della sua mano.
\item De delle informazioni alla difesa.
\item Ammettiamo che il rispondente comunichi di avere un buon numero di punti.
  Allora significa che gli avversari non potrebbero mai fare ostruzioni di nessun tipo.
  Quindi, potreste arrivare a comunicare le stesse informazioni in altro modo.
\item Può dar nascita a dei contro convenzionali per segnalare attacchi più facilmente.
\end{itemize}

\paragraph{Il lato negativo di 2\q}
Se il colore dove siamo sbilanciati fosse \q, allora il nostro compagno ci anticiperebbe il colore.
Facendo giocare alla mano debole il contratto.
Ecco perché si preferisce con quel tipo di mano usare la convenzione 2\q multy.

\paragraph{Esempi pratici:}
Vediamo un esempio dove non si arriva al contratto:
\begin{center}
\begin{auction}
  {}, {2C}, {p}, {2D}, 
  {p}, {2S}, {p}, {p},
  {p}
\end{auction}
\end{center}

\begin{itemize}
\item Nord comunica con 2\f "Ho una mano forte".
\item Sud con 2\q "Descrivi la tua mano"
\item Nord con 2\p "Ho almeno 5 carte di \p".
\item Sud passando fa capire al compagno che non c'è possibilità di fare più di un parziale.
  Potrebbe avere una mano da 0 punti ad esempio.
\end{itemize}

Vediamone uno in cui si arriva a contratto:
\begin{center}
\begin{auction}
  {}, {2C}, {p}, {2D}, 
  {p}, {3C}, {p}, {5C},
  {p}
\end{auction}
\end{center}

\begin{itemize}
\item Nord comunica con 2\f "Ho una mano forte".
\item Sud con 2\q "Descrivi la tua mano"
\item Nord con 3\f "Ho almeno 5 carte di \f".
\item Sud dicendo 5\f comunica appoggio a fiori e punti sufficienti per arrivare a contratto.
\end{itemize}

Vediamo un esempio di dichiarazione per lo slam:

\begin{center}
\begin{auction}
  {}, {2C}, {p}, {2D}, 
  {p}, {2H}, {p}, {3H},
  {p}, {4NT}, {p}, {6H},
  {p}, {p}, {p}
\end{auction}
\end{center}

\begin{itemize}
\item Nord comunica con 2\f "Ho una mano forte".
\item Sud con 2\q "Descrivi la tua mano"
\item Nord con 2\C "Ho almeno 5 carte di \C".
\item Sud dicendo 3\C comunica un interesse per lo slam.
  Se avesse pochi punti passerebbe, tra 2\C+1 e 3\C il punteggio è lo stesso.
  Se avesse solo il punteggio da contratto direbbe direttamente 4\C.
  Dicendo 3\C sta facendo una dichiarazione che indica appoggio, ma di continuare a parlare.
\item Nord in questo esempio chiede direttamente gli assi e Sud manda al piccolo slam.
\end{itemize}


Vediamo un esempio di mano bilanciata: 

\begin{center}
\begin{auction}
  {}, {2C}, {p}, {2D}, 
  {p}, {2NT}, {p}, {3NT},
  {p}, {p}, {p}
\end{auction}
\end{center}

\begin{itemize}
\item Nord comunica con 2\f "Ho una mano forte".
\item Sud con 2\q "Descrivi la tua mano"
\item Nord con 2 NT "Ho una mano bilanciata da 22 punti"
\item Sud decide di chiudere a 3NT per arrivare a contratto. (Probabilite avrà 3-5 punti i mano o poco più)
\end{itemize}




% Apertura 2Q multi
\section{2 Quadri multi}

\subsection{Un aneddoto}
Quando per la prima volta ho giocato la "2 quadri", mi sono ritrovato in un contesto particolare.
Era il mio secondo anno di gioco, e avevo appreso i rudimenti da poco tempo.
Mi ritrovai a giocare con un signore con cui non avevo mai giocato, e in un ambiente in cui non avevo mai giocato.
Ci vedemmo poco prima dell'inizio del torneo per stabilire come giocare.
Questo mi fece una capa tanta sulla necessità di usare la 2\q.
"Tutti la usano qui per competere" mi disse.
Insomma, imparai al volo questa multi memorizzandola e non capendola.
Questo avrebbe dovuto far suonare un campanello d'allarme.

Insomma, durante il torneo spuntò fuori per me una delle mani concordate: "Una monocolore di fiori forte dal almeno 17 punti".
Non ci pensai un momento: 2\q.
Tutto fiero poi mi toccò giocare quella mano.

Poche mani dopo Est apre di 1NT, e io in sud possedendo nuovamente una monocolore minore forte, dichiaro 2\q.
Ovest chiede a Nord(il mio compagno): "Che significato ha? Usate qualche convenzione?".
La sua replica fù: "Non lo abbiamo stabilito, ma direi di no".
Tra me e me pensavo: "Come non lo abbiamo stabilito? Mi hai osannato l'importanza di questa convenzione e ora dici che non la sai?".

La mia ingenuità fu pensare che la convenzione valesse sul apertore della linea, e non della dichiarazione!

\paragraph{Ricordati:}
La convenzione vale solo se sei il primo a licitare qualcosa diverso dal passo.

\begin{center}
\begin{auction}
  {}, {2Q}, {p}, {2H}, 
  {p}, {p}, {p}
\end{auction}
\end{center}
\begin{center}
\begin{auction}
  {p}, {2Q}, {p}, {2S}, 
  {p}, {p}, {p}
\end{auction}
\end{center}

\subsection{Le convenzioni}
Si perché in realtà l'apertura di 2\q in base al sistema e alla coppia significa tutto il contrario di tutto.
Tanto per dirne alcune: Multi color, multi, Messicana, Blue team, Debole, Flannery, Erken...
A maggior ragione, questa proposta è una delle possibili varianti.

\paragraph{Cosa fare davanti alla dichiarazione di 2\q avversaria?}
Inizia con il chiedere cosa significa per loro!
Da li, puoi avere degli accordi di difesa da questa convenzione, in base a quelli decidi.

\paragraph{Mani forte e deboli:}
Nota bene, alcune coppie mettono anche delle mani deboli.
Es. 6 carte di \C/\p con 5-10 punti.

\paragraph{Le mani che ci metto io:}
\begin{itemize}
\item Almeno 6 carte in uno dei minori (\f \q), con 17+ punti.
\item Mano bilanciata da 22 punti
\item Tricolore (4-4-4-1) con almeno 17+ punti.
\end{itemize}

\paragraph{Perché non metto la sesta minore?}
La risposta è presto detta, è una scommessa metterla li.
Vediamo i casi:

\begin{itemize}
\item Possedendo \C
  \begin{itemize} 
  \item Risposta 2\C: Il morto è quello con il palo lungo, aiutiamo la difesa.
  \item Risposta 2\p: Siamo obbligati a parlare a livello 3.
  \end{itemize}

\item Possedendo \p
  \begin{itemize} 
  \item Risposta 2\C:Diciamo 2\p, nessun problema.
  \item Risposta 2\p:Il morto è quello con il palo lungo, aiutiamo la difesa.
  \end{itemize}
\end{itemize}

Può far eccezione se desiderate sfruttare l'apertura di 2\C-\p per una mano con 6 maggiore e 5 minore.
Tuttavia appunto, come mostrato, in 3 casi su 4 facciamo un favore alla difesa, solo in un caso non abbiamo nessun danno.
Ma benefici? Quelli onestamente non ne vedo.
Dichiarare 2\q sbarra certo meno che dichiarare immediatamente 2\C o 2\p.

\subsection{Le risposte}

\paragraph{Cosa si risponde all'apertura di 2\q?}
Le possibilità sono due:
\begin{itemize}
\item Rispondi 2\C con meno di 10 punti.
\item Rispondi 2\p con almeno 10 punti.
\end{itemize}


% Apertura in sbarramento e sotto-apertura.
% Legge della simmetria

% Risposta al compagno

% Intervento 
  % di 1NT
  % in bicolore
  % Sbarramento
  % Costruttivo

% Contro

% Surcontro

% Dichiarazione di sacrificio
% Riaperture, 2NT inusuale. https://www.youtube.com/watch?v=jdcTxB_n7K0

% Legge delle prese totali

% Avvicinarsi a slam
\section{Richiesta per arrivare a slam}

Partiamo con il dire che ne esistono molte di convezioni
per questo scopo.

\paragraph{Lo scopo?}
Capire se sarà possibile con le mani a disposizione
dichiarare un piccolo o grande slam.
Anche avendo tanti punti sulla linea, alla difesa
basta un'asso per vincere contro un contratto da 7x.

\subsection{Richiesta di assi Blackwood}
Tra tutte le richieste di assi questa è quella più semplice.
Per la sua semplicità è anche quella più usata.

\paragraph{Come si innesca la convenzione?}
Dopo aver concordato un atout, uno dei due giocatori dichiara 4NT.

\begin{center}
\begin{auction}
  {}, {1D}, {p}, {1H}, 
  {p}, {3H}, {p}, {4NT}
\end{auction}
\end{center}

\paragraph{E come si risponde?}
Partiamo con il dire: non è possibile non rispondere alla convenzione.
Il passo non è contemplato tra le possibilità.
Anche perché, vedendo l'esempio precedente, avendo trovato un incontro a \C
non possiamo permetterci di lasciare un gioco a senza.

\begin{itemize}
\item 5\f: 0 oppure 4 assi.
\item 5\q: 1 asso.
\item 5\C: 2 assi.
\item 5\p: 3 assi.
\end{itemize}

\subsection{Richiesta assi Roman-Key-Card}
Spesso abbreviata come RKC è una convenzione
che serve a chiedere: "Di quante carte chiave disponi?".

\paragraph{Cosa sono le carte chiave?}
Le carte chiave sono i 4 assi e il re dell'atout concordato.
Nota bene, senza un atout questa convenzione non ha senso!

\paragraph{Come si innesca la convenzione?}
Dopo aver concordato un atout, uno dei due giocatori dichiara 4NT.

\begin{center}
\begin{auction}
  {}, {1D}, {p}, {1H}, 
  {p}, {3H}, {p}, {4NT}
\end{auction}
\end{center}

\paragraph{E come si risponde?}
Partiamo con il dire: non è possibile non rispondere alla convenzione.
Il passo non è contemplato tra le possibilità.
Anche perché, vedendo l'esempio precedente, avendo trovato un incontro a \C
non possiamo permetterci di lasciare un gioco a senza.

\begin{itemize}
\item 5\f: 0 oppure 3 carte chiave.
\item 5\q: 1 oppure 4 carte chiave.
\item 5\C: 2 carte chiave e non si possiede la Q dell'atout concordato.
\item 5\p: 2 carte chiave e si possiede la Q dell'atout concordato.
\end{itemize}

Queste sono le risposte della convenzione di base, tuttavia possiamo fare
un ulteriore passo.
\begin{itemize}
\item 5NT: Numero pari di carte chiave e un vuoto in uno dei semi.
\item 6\f: Numero dispari di carte chiave.
  Se il seme di atout fosse \f si sta dicendo che si ha un vuoto non specificato.
  Se il seme di atout non fosse \f, si sta dicendo di essere vuoti a \f.
\item 6\q: Numero dispari di carte chiave.
  Se il seme di atout fosse \q si sta dicendo che si ha un vuoto a \C o \p.
  Se il seme di atout fosse \C o \p, si sta dicendo di essere vuoti a \q.

\item 6\C:Numero dispari di carte chiave.
  Se il seme di atout fosse \C si sta dicendo che si ha un vuoto a \p.
  Se il seme di atout non fosse \p, si sta dicendo di essere vuoti a \C.
\end{itemize}

Facciamo qualche considerazione rispetto a questo secondo blocco di risposte:
\begin{itemize}
\item Non andiamo mai sopra il livello di atout.
Es. Atout con \C non dichiareremo mai 6\p.
oppure, atout \q, non dichiareremo mai ne 6\C ne 6\p.

\item Se il palo citato non è atout, si sta solitamente dicendo
che si ha un vuoto in quel palo.
\end{itemize}

\paragraph{Dichiarazione implicita o esplicita di atout}
Ci sono delle situazioni in cui vorremmo comunicare
il meno possibile per evitare di dare informazioni alla difesa.

Esplicitare di avere un incontro in un seme può diventare
un'operazione che "regala" informazioni.
Vediamo un esempio:

\begin{center}
\begin{auction}
  {}, {1C}, {p}, {1H}, 
  {p}, {3H}, {p}, {4NT}
\end{auction}
\end{center}

In questa licita il colore di atout si dice esplicito.
I due giocatori si sono incontrati avendo entrambi le cuori.
Vediamo una seconda mano in cui invece l'incontro sarà implicito:

\begin{center}
\begin{auction}
  {}, {1C}, {p}, {1H}, 
  {p}, {4NT}, {p}
\end{auction}
\end{center}

In questo caso, il rispondete alla richiesta di assi sa
che l'atout concordato è \C senza però dare ulteriori informazioni
sulla propria mano.

\paragraph{Quando preferiamo fare richieste esplicite e quando no?}
Se abbiamo già un'idea chiara dei punti sulla linea non serve
dare ulteriori informazioni alla difesa.
Viceversa, se le mani dichiarate non sono limpide,
è meglio esplicitare le informazioni.

\paragraph{Ma non c'è ambiguità sul colore di atout?}
No, non c'è mai ambiguità.
Quando citiamo un seme e vediamo come risposta la richiesta di assi,
sappiamo che chi gioca con noi vuole quel seme come atout.

\paragraph{Richiesta di regina atout}
Le prime due risposte del primo livello: 5\f, 5\q.
Non specificano la presenza o assenza di regina dell'atout concordato.
A differenza delle altre risposte del primo livello: 5\C, 5\p.

Qualcuno potrebbe non valutare questo come un dato rilevante.
Tuttavia, se in alcune mani questo dato può anche risultare tale,
spesso invece ci è utile sapere se la regina è in nostro possesso.
Ecco che viene in soccorso una continuazione diversa dalla richiesta di re.

Il giocatore che ha chiesto gli assi, davanti ad una risposta di 5\f o 5\q,
licita il primo seme che non sia l'atout.
Cosa significa questa licita? "Potresti dirmi se possiedi la regina?".

Le risposte sono le seguenti:
\begin{itemize}
\item Tornare nel colore di atout indica assenza della regina.
\item 5NT: "Ho la regina di atout e non posseggo dei re".
\item 6x: "Ho la regina di atout e il re del seme x".
\end{itemize}

\subsection{Richiesta di Re}
\paragraph{Premesse necessarie}
Chiediamo i re se sappiamo di avere il controllo di tutti gli assi.
O come minimo, dove manca l'asso siamo sicuri di non perdere nessuna presa.
Possiamo essere certi di questa cosa solo se in una delle due mani
c'è un vuoto in quel seme.

\paragraph{Richiesta dei re in relazione con gli assi}
In base al tipo di richiesta fatta precedentemente con gli assi,
la richiesta di re potrebbe variare.
Perché? Se usate una convenzione come RKC avete già segnalato uno dei re.
Attenzione a non ripetere in questo caso una carta già detta.

\paragraph{Versione classica}
Uno dei due giocatori gioca 5NT che significa:
"Dimmi quanti re possiedi".

Il rispondete ora è obbligato a rispondere alla convenzione con
una delle seguenti risposte:
\begin{itemize}
\item 6\f: Ho 0 oppure 4 re.
\item 6\q: Ho 1 re.
\item 6\C: Ho 2 re.
\item 6\p: Ho 3 re.
\end{itemize}

\paragraph{Versione ridotta avendo già comunicato un re}
Uno dei due giocatori gioca 5NT che significa:
"Dimmi quanti re possiedi".

Il rispondete ora è obbligato a rispondere alla convenzione con
una delle seguenti risposte:
\begin{itemize}
\item 6\f: Ho 0 re.
\item 6\q: Ho 1 re.
\item 6\C: Ho 2 re.
\item 6\p: Ho 3 re.
\end{itemize}

% Cuebid

\newpage
