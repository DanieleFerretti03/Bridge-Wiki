\section{2 Quadri multi}

\subsection{Un aneddoto}
Quando per la prima volta ho giocato la "2 quadri", mi sono ritrovato in un contesto particolare.
Era il mio secondo anno di gioco, e avevo appreso i rudimenti da poco tempo.
Mi ritrovai a giocare con un signore con cui non avevo mai giocato, e in un ambiente in cui non avevo mai giocato.
Ci vedemmo poco prima dell'inizio del torneo per stabilire come giocare.
Questo mi fece una capa tanta sulla necessità di usare la 2\q.
"Tutti la usano qui per competere" mi disse.
Insomma, imparai al volo questa multi memorizzandola e non capendola.
Questo avrebbe dovuto far suonare un campanello d'allarme.

Insomma, durante il torneo spuntò fuori per me una delle mani concordate: "Una monocolore di fiori forte dal almeno 17 punti".
Non ci pensai un momento: 2\q.
Tutto fiero poi mi toccò giocare quella mano.

Poche mani dopo Est apre di 1NT, e io in sud possedendo nuovamente una monocolore minore forte, dichiaro 2\q.
Ovest chiede a Nord(il mio compagno): "Che significato ha? Usate qualche convenzione?".
La sua replica fù: "Non lo abbiamo stabilito, ma direi di no".
Tra me e me pensavo: "Come non lo abbiamo stabilito? Mi hai osannato l'importanza di questa convenzione e ora dici che non la sai?".

La mia ingenuità fu pensare che la convenzione valesse sul apertore della linea, e non della dichiarazione!

\paragraph{Ricordati:}
La convenzione vale solo se sei il primo a licitare qualcosa diverso dal passo.

\begin{center}
\begin{auction}
  {}, {2Q}, {p}, {2H}, 
  {p}, {p}, {p}
\end{auction}
\end{center}
\begin{center}
\begin{auction}
  {p}, {2Q}, {p}, {2S}, 
  {p}, {p}, {p}
\end{auction}
\end{center}

\subsection{Le convenzioni}
Si perché in realtà l'apertura di 2\q in base al sistema e alla coppia significa tutto il contrario di tutto.
Tanto per dirne alcune: Multi color, multi, Messicana, Blue team, Debole, Flannery, Erken...
A maggior ragione, questa proposta è una delle possibili varianti.

\paragraph{Cosa fare davanti alla dichiarazione di 2\q avversaria?}
Inizia con il chiedere cosa significa per loro!
Da li, puoi avere degli accordi di difesa da questa convenzione, in base a quelli decidi.

\paragraph{Mani forte e deboli:}
Nota bene, alcune coppie mettono anche delle mani deboli.
Es. 6 carte di \C/\p con 5-10 punti.

\paragraph{Le mani che ci metto io:}
\begin{itemize}
\item Almeno 6 carte in uno dei minori (\f \q), con 17+ punti.
\item Mano bilanciata da 22 punti
\item Tricolore (4-4-4-1) con almeno 17+ punti.
\end{itemize}

\paragraph{Perché non metto la sesta minore?}
La risposta è presto detta, è una scommessa metterla li.
Vediamo i casi:

\begin{itemize}
\item Possedendo \C
  \begin{itemize} 
  \item Risposta 2\C: Il morto è quello con il palo lungo, aiutiamo la difesa.
  \item Risposta 2\p: Siamo obbligati a parlare a livello 3.
  \end{itemize}

\item Possedendo \p
  \begin{itemize} 
  \item Risposta 2\C:Diciamo 2\p, nessun problema.
  \item Risposta 2\p:Il morto è quello con il palo lungo, aiutiamo la difesa.
  \end{itemize}
\end{itemize}

Può far eccezione se desiderate sfruttare l'apertura di 2\C-\p per una mano con 6 maggiore e 5 minore.
Tuttavia appunto, come mostrato, in 3 casi su 4 facciamo un favore alla difesa, solo in un caso non abbiamo nessun danno.
Ma benefici? Quelli onestamente non ne vedo.
Dichiarare 2\q sbarra certo meno che dichiarare immediatamente 2\C o 2\p.

\subsection{Le risposte}

\paragraph{Cosa si risponde all'apertura di 2\q?}
Le possibilità sono due:
\begin{itemize}
\item Rispondi 2\C con meno di 10 punti.
\item Rispondi 2\p con almeno 10 punti.
\end{itemize}
