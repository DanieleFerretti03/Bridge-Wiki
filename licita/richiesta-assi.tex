\section{Richiesta per arrivare a slam}

Partiamo con il dire che ne esistono molte di convezioni
per questo scopo.

\paragraph{Lo scopo?}
Capire se sarà possibile con le mani a disposizione
dichiarare un piccolo o grande slam.
Anche avendo tanti punti sulla linea, alla difesa
basta un'asso per vincere contro un contratto da 7x.

\subsection{Richiesta di assi Blackwood}
Tra tutte le richieste di assi questa è quella più semplice.
Per la sua semplicità è anche quella più usata.

\paragraph{Come si innesca la convenzione?}
Dopo aver concordato un atout, uno dei due giocatori dichiara 4NT.

\begin{center}
\begin{auction}
  {}, {1D}, {p}, {1H}, 
  {p}, {3H}, {p}, {4NT}
\end{auction}
\end{center}

\paragraph{E come si risponde?}
Partiamo con il dire: non è possibile non rispondere alla convenzione.
Il passo non è contemplato tra le possibilità.
Anche perché, vedendo l'esempio precedente, avendo trovato un incontro a \C
non possiamo permetterci di lasciare un gioco a senza.

\begin{itemize}
\item 5\f: 0 oppure 4 assi.
\item 5\q: 1 asso.
\item 5\C: 2 assi.
\item 5\p: 3 assi.
\end{itemize}

\subsection{Richiesta assi Roman-Key-Card}
Spesso abbreviata come RKC è una convenzione
che serve a chiedere: "Di quante carte chiave disponi?".

\paragraph{Cosa sono le carte chiave?}
Le carte chiave sono i 4 assi e il re dell'atout concordato.
Nota bene, senza un atout questa convenzione non ha senso!

\paragraph{Come si innesca la convenzione?}
Dopo aver concordato un atout, uno dei due giocatori dichiara 4NT.

\begin{center}
\begin{auction}
  {}, {1D}, {p}, {1H}, 
  {p}, {3H}, {p}, {4NT}
\end{auction}
\end{center}

\paragraph{E come si risponde?}
Partiamo con il dire: non è possibile non rispondere alla convenzione.
Il passo non è contemplato tra le possibilità.
Anche perché, vedendo l'esempio precedente, avendo trovato un incontro a \C
non possiamo permetterci di lasciare un gioco a senza.

\begin{itemize}
\item 5\f: 0 oppure 3 carte chiave.
\item 5\q: 1 oppure 4 carte chiave.
\item 5\C: 2 carte chiave e non si possiede la Q dell'atout concordato.
\item 5\p: 2 carte chiave e si possiede la Q dell'atout concordato.
\end{itemize}

Queste sono le risposte della convenzione di base, tuttavia possiamo fare
un ulteriore passo.
\begin{itemize}
\item 5NT: Numero pari di carte chiave e un vuoto in uno dei semi.
\item 6\f: Numero dispari di carte chiave.
  Se il seme di atout fosse \f si sta dicendo che si ha un vuoto non specificato.
  Se il seme di atout non fosse \f, si sta dicendo di essere vuoti a \f.
\item 6\q: Numero dispari di carte chiave.
  Se il seme di atout fosse \q si sta dicendo che si ha un vuoto a \C o \p.
  Se il seme di atout fosse \C o \p, si sta dicendo di essere vuoti a \q.

\item 6\C:Numero dispari di carte chiave.
  Se il seme di atout fosse \C si sta dicendo che si ha un vuoto a \p.
  Se il seme di atout non fosse \p, si sta dicendo di essere vuoti a \C.
\end{itemize}

Facciamo qualche considerazione rispetto a questo secondo blocco di risposte:
\begin{itemize}
\item Non andiamo mai sopra il livello di atout.
Es. Atout con \C non dichiareremo mai 6\p.
oppure, atout \q, non dichiareremo mai ne 6\C ne 6\p.

\item Se il palo citato non è atout, si sta solitamente dicendo
che si ha un vuoto in quel palo.
\end{itemize}

\paragraph{Dichiarazione implicita o esplicita di atout}
Ci sono delle situazioni in cui vorremmo comunicare
il meno possibile per evitare di dare informazioni alla difesa.

Esplicitare di avere un incontro in un seme può diventare
un'operazione che "regala" informazioni.
Vediamo un esempio:

\begin{center}
\begin{auction}
  {}, {1C}, {p}, {1H}, 
  {p}, {3H}, {p}, {4NT}
\end{auction}
\end{center}

In questa licita il colore di atout si dice esplicito.
I due giocatori si sono incontrati avendo entrambi le cuori.
Vediamo una seconda mano in cui invece l'incontro sarà implicito:

\begin{center}
\begin{auction}
  {}, {1C}, {p}, {1H}, 
  {p}, {4NT}, {p}
\end{auction}
\end{center}

In questo caso, il rispondete alla richiesta di assi sa
che l'atout concordato è \C senza però dare ulteriori informazioni
sulla propria mano.

\paragraph{Quando preferiamo fare richieste esplicite e quando no?}
Se abbiamo già un'idea chiara dei punti sulla linea non serve
dare ulteriori informazioni alla difesa.
Viceversa, se le mani dichiarate non sono limpide,
è meglio esplicitare le informazioni.

\paragraph{Ma non c'è ambiguità sul colore di atout?}
No, non c'è mai ambiguità.
Quando citiamo un seme e vediamo come risposta la richiesta di assi,
sappiamo che chi gioca con noi vuole quel seme come atout.

\paragraph{Richiesta di regina atout}
Le prime due risposte del primo livello: 5\f, 5\q.
Non specificano la presenza o assenza di regina dell'atout concordato.
A differenza delle altre risposte del primo livello: 5\C, 5\p.

Qualcuno potrebbe non valutare questo come un dato rilevante.
Tuttavia, se in alcune mani questo dato può anche risultare tale,
spesso invece ci è utile sapere se la regina è in nostro possesso.
Ecco che viene in soccorso una continuazione diversa dalla richiesta di re.

Il giocatore che ha chiesto gli assi, davanti ad una risposta di 5\f o 5\q,
licita il primo seme che non sia l'atout.
Cosa significa questa licita? "Potresti dirmi se possiedi la regina?".

Le risposte sono le seguenti:
\begin{itemize}
\item Tornare nel colore di atout indica assenza della regina.
\item 5NT: "Ho la regina di atout e non posseggo dei re".
\item 6x: "Ho la regina di atout e il re del seme x".
\end{itemize}

\subsection{Richiesta di Re}
\paragraph{Premesse necessarie}
Chiediamo i re se sappiamo di avere il controllo di tutti gli assi.
O come minimo, dove manca l'asso siamo sicuri di non perdere nessuna presa.
Possiamo essere certi di questa cosa solo se in una delle due mani
c'è un vuoto in quel seme.

\paragraph{Richiesta dei re in relazione con gli assi}
In base al tipo di richiesta fatta precedentemente con gli assi,
la richiesta di re potrebbe variare.
Perché? Se usate una convenzione come RKC avete già segnalato uno dei re.
Attenzione a non ripetere in questo caso una carta già detta.

\paragraph{Versione classica}
Uno dei due giocatori gioca 5NT che significa:
"Dimmi quanti re possiedi".

Il rispondete ora è obbligato a rispondere alla convenzione con
una delle seguenti risposte:
\begin{itemize}
\item 6\f: Ho 0 oppure 4 re.
\item 6\q: Ho 1 re.
\item 6\C: Ho 2 re.
\item 6\p: Ho 3 re.
\end{itemize}

\paragraph{Versione ridotta avendo già comunicato un re}
Uno dei due giocatori gioca 5NT che significa:
"Dimmi quanti re possiedi".

Il rispondete ora è obbligato a rispondere alla convenzione con
una delle seguenti risposte:
\begin{itemize}
\item 6\f: Ho 0 re.
\item 6\q: Ho 1 re.
\item 6\C: Ho 2 re.
\item 6\p: Ho 3 re.
\end{itemize}
