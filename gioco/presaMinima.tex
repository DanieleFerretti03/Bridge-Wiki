\section{La carta con cui prendere}

Sia in difesa che in attacco ci ritroviamo con la domanda "con che carta prendo?".
Trattiamo di seguito un principio essenziale per la difesa, e un trucchetto utile per il dichiarante.

\paragraph{Il dichiarante:}
Supponiamo che in mano vi rimanga un A\p e stiate giocando a senza atout.
Farete la presa oppure no? Naturalmente la risposta è si.

Ora, stessa situazione, ma la carta è il K\p.
Ripropongo la stessa domanda, ma in questo caso la risposta è: "dipende".
Dipende da cosa? Se l'asso è stato giocato oppure no.

Nel caso specifico, avendo a disposizione solo una carta la scelta non si pone, quella va giocata.
Supponiamo però per un momento che non sia l'ultima, e che voi dobbiate stabilire se una certa carta è franca o meno.

Si sa, in una mano di Bridge sono tanti i fattori da tenere a mente.
Ecco perché cerchiamo di applicare delle strategie per ridurre questo carico di lavoro.

Avendo in mano: AKQJ10, non prendete mai di A, ma anzi, usate il 10.

Questa operazione ha due scopi:
\begin{itemize}
\item Il primo è quello citato prima.
  Se non ricordassimo se le carte sono buone o meno, avere un asso è diverso che avere un dieci in mano.

\item La seconda, è quella di dare meno informazioni alla difesa. Lasciandoli con il dubbio che l'altra persona in difesa possegga qualche onore.
\end{itemize}

\paragraph{E in difesa come posso sfruttare la cosa?}

Supponiamo che ovest attacchi con il 3\f.
Est copra con A\f avendo a disposizione AKQ\f.
Le informazioni che arrivano alla difesa sono inesistenti.

Ora, ipotizziamo che invece est copra con Q\f, e che sud non prenda.
Gli scenari che si aprono son due:
\begin{itemize}
\item Sud non ha carte più alte della Q\f per coprire
\item Sud potrebbe avere delle carte più alte della Q\f ma ha bisogno di tenere i fermi.
\end{itemize}

In entrambi i casi, abbiamo ottenuto delle informazioni utili!

\paragraph{Avere questo modo di giocare ha altre implicazioni?}
Certo, mettere la carta minima potrebbe aiutare in alcune situazioni a fare delle prese in più.

