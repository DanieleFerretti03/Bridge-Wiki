\section{Divisioni delle carte}

Con divisioni intendiamo il numero di carte presenti tra due mani.
Sappiamo ottenere il numero totale di carte tra due mani conoscendo il numero di carte nelle altre due.

Es. Una persona gioca in difesa e ha 5 carte di \C, vede al morto che ci sono altre 5 carte di \C.
Sapendo che in totale le carte di cuori sono 13, sappiamo che tra le due mani il numero di carte rimaste è 3.

\paragraph{Come mi aiuta sapere questa cosa?}
Per comprendere meglio i meccanismi della probabilità sul gioco della carta.

\paragraph{L'aforisma:}
Se il numero è pari, la divisione più più probabile è $\frac{n}{2}$.
Es. Mancando 8 carte è più probabile siano divise 4-4.

Se il numero è dispari la cosa più probabile è avere i numeri vicini alla metà separati di un'unità.
Es. Mancando 5 carte è più probabile la dizione 3-2 o 2-3.

\paragraph{Attenzione alla dichiarazione:}
Se una persona dichiara di avere un certo numero di carte, è evidente che dovremmo aggiustare i nostri ragionamenti di conseguenza.
Se Nord dichiara di avere 6 carte di cuori, e tra sud-nord mancano 8 carte, la divisione "certa" sarà 6-2 e non la statisticamente probabile 4-4.

\paragraph{Come posso calcolare queste possibilità?}
Nota, al tavolo da gioco non serve che ti metti a fare un conto matematico.
Tuttavia, per avere un numero approssimativo potresti fare così:

Dove tot indica il numero di carte mancanti, e m1+m2=tot.
M1 e m2 indicano le distribuzioni possibili.
$\frac{tot!}{m1!*m2!}$

Il numero che esce fuori va comparato con altre distribuzioni, il numero più alto è quello più probabile.
Nota '!' indica l'operazione di fattoriale.
