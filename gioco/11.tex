\section{Legge 10, 11 e 12}

Questa "legge" è basata su di un principio matematico.
È bene tenerla a mente e farvi pratica.
Essa può essere usata sia dal compagno in difesa di chi attacca, sia dal dichiarante.
Il momento in cui si potrà applicare è quello dell'attacco.

Supponiamo che la mano giocata sia:

\board{}{7\p}%
{K52,--,--,--}%
{AQJ,--,--,--}%
{643,--,--,--}%
{78910,--,--,--}

Supponiamo che il contratto sia giocato da sud.
L'attacco da parte di ovest è fatto con la sua 4 carta migliore: in questo caso il 7\p.

\paragraph{Nei panni della difesa:}
Supponiamo di essere nei panni di est, e quindi di avere AQJ\p.
Sapendo che il nostro compagno ha attaccato con la sua quarta migliore, sottraiamo da 11 il valore ella carta di attacco.
In questo caso quindi $11-7=4$, queste sono il numero di carte più alte del (in questo caso) 7\p nelle altre tre mani.

Per rendere questo dato più utile, vedendo il morto possiamo sottrarre ulteriormente una carta più alta (in questo caso il K\p), quindi $11-4-1=3$.
Ovvero tra la mano di est e quella di sud ci sono 3 carte più alte del 7\p.

Possiamo ulteriormente raffinare questo numero, siccome est vede ovviamente le sue carte.
In questo caso possiamo sottrarre 3 carte (AQJ\p), quindi $11-4-1-3 = 0$.
Questo 0 rappresenta il numero di carte più alte del 7 nella mano di sud.
Ovvero ha solo carte più basse del 7, vedendo le carte potete constatare la veridicità dell'affermazione.

\paragraph{Nei panni del dichiarante:}
Usando sempre la mano di prima.
Sottraiamo da il numero della carte di attacco $11-7=4$.
Raffiniamo eliminando le carte del morto $11-7-1=3$.
Concludiamo con l'eliminare il numero di carte più alte del 7\p nella nostra mano.
In questo caso, nessuna.
Quindi, sappiamo che nella mano di est ci sono 3 carte più alte del 7.

Nuovamente, guardate la distribuzione e confermate la veridicità dell'affermazione.

\paragraph{E se non attaccasse con la 4 migliore?}
Sicuramente non si potrebbe applicare la legge 11.
Tuttavia, non tutto è ancora perduto!
Attaccando con la terza migliore, si può applicare la legge del 12.
La differenza tra le due nei conti è solo nel sostituire il numero 11 al numero 12.

Facciamo un esempio pratico:

\board{}{5\f}%
{--,--,--,AQJ7}%
{--,--,--,98}%
{--,--,--,643}%
{--,--,--,K1052}

\begin{itemize}

\item Per la difesa: $12-5-4-2=1$.
  12 è il numero di base, 5\f è la carta di attacco,
  4 il numero di carte più alte possedute dal nord (AQJ7\f),
  2 il numero di carte più alte possedute da est (98\f).
  Se si nota, sud ha solo una carta superiore al 5\f (ovvero il 6\f).

\item Per il dichiarante: $12-5-4-1=2$.
  12 è il numero di base, 5\f è la carta di attacco,
  4 il numero di carte più alte possedute dal nord (AQJ7\f),
  1 il numero di carte più alte possedute da sud (6\f).
  Se si nota, est ha solo due carta superiore al 5\f (ovvero il 9 e 8 di \f).
\end{itemize}

\paragraph{E se l'attacco fosse la quinta migliore?}
Stesso discorso che abbiamo fatto con la legge 11 e del 12.
In questo frangente però il numero di base sarà il 10.

\paragraph{E se non attaccasse ne con la quarta ne con la terza e nemmeno con la quinta migliore?}
Nulla da fare, saranno altre le considerazioni da dover fare.
Certo, non si potrà usare la legge 10, 11 o 12.

\paragraph{Quando è meglio non attaccare in questo modo?}
Sarebbe meglio evitare questa cosa se vi trovate come 5a, 4a o 3a migliore con un numero molto basso.
Sapere che in mano al dichiarante ci sono 3 carte più alte del 2 non è molto utile per la difesa, o no?
