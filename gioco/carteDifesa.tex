\section{Le carte della difesa}
Premettiamo che esistono diversi sistemi per contare il numero di carte delle difesa.
Io propongo quello che mi sembra più facile da applicare.

\paragraph{A cosa serve contare le carte della difesa?}
Inanzi tutto, in un gioco a NT ci dà l'idea se delle scartine sono diventate franche o meno.
In un gioco a colore è utile sapere se tutti gli atout sono usciti di scena.

\paragraph{Perché esistono tanti sistemi?}
Ogni giocatore usa quello con cui si trova meglio, e logicamente quello che funziona per me non funziona per altri.

\paragraph{È un'operazione difficile?}
Non devi certo ricordarti un numero astronomico di informazioni.
Facendo pratica in ogni caso è più facile con il tempo migliorare.
All'inizio può sembrare ostico, ma vedrete che con il tempo lo prenderete come automatismo.

\paragraph{Quale sistema si può adottare quindi?}
Prendiamo il numero di carte che vediamo dal morto e nella nostra mano.
Sottraiamo quel numero a 13 (le carte in un seme), ed ecco la risposta di quante carte sono rimaste in mano agli avversari.

Di fatto, quando giochiamo, non ci interessa tenere a mente altre informazioni.

\paragraph{Mi sembra difficile:}
Certo, detta così è difficile! 
Il segreto sta nel contare il numero di carte in un certo seme alla volta.
Prendiamo il gioco a colore, prima pensiamo al atout e dopo agli alti semi.

\paragraph{Durante il gioco cosa devo fare?}
Se stai giocando il seme di \q, e vedi 8 carte tra te e il morto sai che gli avversari avranno 5 carte di \q.
Facendo una presa a \q, se entrami rispondono scartando una carta di \q, aggiorni quel numero $5-2=3$.
Solitamente diminuisci quel numero di 2 in 2, tuttavia, fai attenzione agli avversari che non rispondono.

Se al secondo giro di \q, solo uno dei due scarta \q e l'altro scarta \f, saprai che uno possiede ancora 2 carte di \q.
A quel punto, potrai tenere a mente anche quante sono il numero di fiori in possesso agli avversari.

\paragraph{Alla fine dovrò comunque tenere a mente tutti i numeri quindi?}
No certo che no.
Una volta che in un certo palo gli avversari non hanno più carte quel numero non lo dovrai più tenere a mente.
Potrebbe anche interessarti sapere che non serve necessariamente memorizzare tutti i semi.
Avendo ad esempio il palo di \p che sappiamo essere perdente, non ci interessa sapere quante carte gli avversari hanno in mano.
Memorizza il numero di carte della difesa solo quando serve.

