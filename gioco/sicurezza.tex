\section{Gioco in sicurezza}

Questo stile di gioco si applica principalmente a senza.
Tuttavia, alcune volte serve applicarlo anche in un gioco a colore.

\paragraph{Come funziona?}
Si tratta di fare fin da subito tutte le prese che si hanno a disposizione.
Non lasciandosi tentare da qualche azzardo per aumentare il numero di prese totali.

\paragraph{Quando applicare il gioco in sicurezza?}
Per stabilirlo basta fare questa considerazione.
Se gli avversari prendessero una presa, giocando le loro carte, rischierebbero di farci fallire il contratto?
Se la risposta è si, o quasi sicuramente, allora è il caso fare tutte le prese il prima possibile.

Vediamo un esempio di una situazione pratica:
Sud gioca 1NT e attacca ovest con il A\p.
Scarti due scartine ed est mette il 9, Ovest, torna a picche nuovamente ma questa volta prendi con il re.
\board{}{}%
{K42,54,QJ63,AK97}%
{QJ109,KJ982,1087,QJ10}%
{865,10763,AK4,852}%
{A73,AQ92,952,643}

In questa situazione, non possiamo permettere di tornare in mano agli avversari.
Di cuori non abbiamo nessun fermo, e se tornassero a picche si farebbero le ultime rimaste in mano senza problemi.

Prendendo di K\p possiamo incassare 4 prese di \q e 2 prese di \f.
Sommandole alla presa iniziale di \p arriviamo a 7 prese per rispettare il contratto.

Diversamente, se non avessimo fatto tutte le prese fin da subito gli avversari avrebbero potuto incassare facilmente 7 prese.
Facendo così fallire il contratto.

\paragraph{Anche a colore si può giocare in sicurezza?}
Certo come avevamo detto prima.
Anche in quelle situazioni potremmo avere delle carte perdenti che non possiamo scartare.

\paragraph{Non è bene giocare sempre in sicurezza?}
Assolutamente no!
In un torneo, se tutta la sala chiama 5\f e noi segniamo 5\f+1 abbiamo fatto un top.
Giocando solo le prese certe non state facendo nulla di strano.
Quelle prese in ogni caso vi sono dovute per definizione.
Se non rischiate di perdere il contratto, il rischio di lasciare qualche presa agli avversari potrebbe ricompensarvi con un maggior numero di prese.

\paragraph{Giocare bene vs giocare in sicurezza}
Non bisogna confondere il giocare bene e giocare in sicurezza.
Chi gioca in sicurezza applica per definizione le buone pratiche del gioco.
Chi gioca bene non per forza si dispone per giocare in sicurezza.

In questa combinazione, incassiamo prima l'asso e poi tentiamo l'empass al Q.
\begin{center}
\begin{tabular}{|l|}
J109876              \\
\hline
AK32 
\end{tabular}
\end{center}
Giocare in questo modo ci da le migliori possibilità di riuscita, ma non si può definire gioco in sicurezza!

Diversa è questa situazione:
\begin{center}
\begin{tabular}{|l|}
AKQ10xx\\
\hline
x 
\end{tabular}
\end{center}

In questa situazione, un gioco in sicurezza suggerirebbe di incassare i tre onori e sperare di vedere J cadere.
Facendo così avremmo sicuramente tre prese, e non metteremo certamente in mano agli avversari.
Se però, non abbiamo il timore di mettere in mano gli avversari, il miglior modo per assicurarci 4 prese è quello di tentare l'empass con il 10 e poi incassare gli onori.

In questa situazione si distingue bene cosa rappresenta giocare in sicurezza e cosa no.
